
\chapter{Увод}
\section{Идеја}
Пројектни задатак, као идеја, је започет покушајем имитирања тенкова који беже од противтенковских ракета. Тенк је симулиран као аутић са 4 точка који се помера на електрични погон уз помоћ алкалних батерија. Док је ракета симулирана светлошћу која му се приближава или удаљава. Ради лакшег тестирања и дебаговања, коришћена је Windows Mobile апликација \cite{BluetoothTerminal} која преко bluetooth модула прима одговарајуће информације од уређаја (јачина тренутно детектоване светлости,...) . 

\section{Функционалности} \label{sec:fun}
У пројекту су подржани следећи типови функционалности:
\begin{enumerate}[1.]
\item \textbf{Тражење светлости} 
\\ Кад се аутић налази у овом моду, он обилази круг фиксирањем једног точка, док се други окреће одређеном брзином. Ово резултује ротирањем аутића око једног точка. Он својом логиком детектује најјачи извор светлости коришћењем фотосензора за детекцију јачине светлости, и памти као позицију на кругу који је обишао.
\item \textbf{Лоцирање светлости}
\\ Аутић прелази у овај мод кад обиђе круг у претходном. Сад се покушава позиционирати од најјачег детектованог извора светлости (одређеној тачки на кругу). Позиционирање се обавља као у моду тражења светлости из угла померања точкова.  Обилази се део круга тако да на крају предњи део аутића буде позициониран од најјачег извора светлости.
\item \textbf{Бежање од светлости}
\\ Како се позиционирао на одговарајућу тачку круга на којој је у моду тражења светлости детектовао најјачи извор светлости, оба точка се окрећу напред \footnote{Одрађено на изложени начин јер није могуће точкове померати уназад} истим брзинама и беже одређено фиксно време од извора светлости. Кад заврши бежање прелази се на први мод.
\item \textbf{Слање података Windows mobile апликацији}
\\ Сваких 50 ms поред израчунавања везаних за детекцију јачине светлости на аутићу, обавља се и слање одговарајућих података UART протоколом  преко Bluetooth конекције упареном Bluetooth уређају. Одговарајући подаци који се шаљу су тренутно детектована јачина светлости, максимално детектована јачина светлости, тренутна позиција (колико је остало до краја обиласка у круг за мод тражења светлости, до краја лоцирања светлости колико је остало и до краја "бежања" од светлости за мод бежања светлости), као и који је тренутни мод од одговарајућа три. На крају слања се шаље и -1 да би корисник bluetooth апликације знао да је послат читав скуп података са аутића.  

\end{enumerate}